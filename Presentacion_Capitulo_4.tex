\documentclass{beamer}

\usetheme{Madrid}
\usecolortheme{default}
\usefonttheme{professionalfonts}
\useinnertheme{rounded}

% --- PREÁMBULO ---
\usepackage[utf8]{inputenc}
\usepackage[T1]{fontenc}
\usepackage[spanish]{babel}
\usepackage{amsmath, amssymb}
\usepackage{graphicx}
\usepackage{tikz}
\usetikzlibrary{positioning, calc, arrows.meta, shapes, arrows, shapes.geometric, fit}
\usepackage{pgfplots}
\pgfplotsset{compat=1.18}
\usepackage{tcolorbox}
\usepackage{ragged2e}
\usepackage{array}
\usepackage{booktabs}
\usepackage{multicol}
\usepackage{caption} % Para poder usar captions dentro de figure
\usepackage{media9}
\usepackage{listings}
\usepackage{xcolor}
\usepackage{hyperref}

\newcommand{\figplaceholder}[2][0.6\textwidth]{%
    \begin{tikzpicture}
        \node[draw, dashed,
              minimum width=#1,
              text width=0.9*#1,
              minimum height=0.5*#1,
              align=center,
              text=gray,
              inner sep=3pt
              ] {#2};
    \end{tikzpicture}%
}

% --- METADATOS DE LA PRESENTACIÓN ---
\title{Transformaciones de Intensidad y Filtrado Espacial}
\subtitle{Basado en el Cap. 3 de ``Digital Image Processing'' \\ (Gonzalez \& Woods)}
\author{Nombre del Presentador  \small{(Adaptado de material técnico)}}
\institute{Universidad / Institución}
\date{\today}

\begin{document}

% --- DIAPOSITIVA DE TÍTULO ---
\begin{frame}
    \titlepage
\end{frame}

% --- DIAPOSITIVA DE CONTENIDO (TABLA DE MATERIAS) ---
\begin{frame}
    \frametitle{Contenido}
    \tableofcontents
\end{frame}

% --- SECCIÓN 1 ---
\section{Introducción y Conceptos Fundamentales}

\begin{frame}[fragile]{Antecedentes}

\href{run:Antecedentes.html}{4 Antecedentes}

\end{frame}



\begin{frame}{Preview del Capítulo}
    \begin{block}{Dominio Espacial}
        El término \textit{dominio espacial} se refiere al plano mismo de la imagen, y los métodos de procesamiento en esta categoría se basan en la manipulación directa de los píxeles en una imagen.
    \end{block}
    \begin{itemize}
        \item Se contrasta con el procesamiento en el \textit{dominio de la frecuencia} (o transformado), donde la imagen primero se transforma, luego se procesa, y finalmente se aplica la transformada inversa.
        \item Dos categorías principales de procesamiento en el dominio espacial:
        \begin{itemize}
            \item \textbf{Transformaciones de intensidad}: Operan en píxeles individuales. Útiles para manipulación de contraste y segmentación por umbralización.
            \item \textbf{Filtrado espacial}: Realiza operaciones sobre la vecindad de cada píxel. Útil para suavizado y realce de bordes.
        \end{itemize}
    \end{itemize}
\end{frame}



%=================================================================
% Sección 9: Últimos Recursos y Enlaces
%=================================================================
\section{Recursos Adicionales}

\begin{frame}[fragile]{Enlaces a Contenido Complementario}
    \begin{itemize}
        \item \href{run:Tema_pag_150_p1.html}{\textbf{3.4 Fundamentos del Filtrado Espacial}}
        \item \href{run:Tema_3_4_pag_154_p1.html}{\textbf{Correlación Espacial y Convolución}}
        \item \href{run:Tema_3_4_pag_161_p1.html}{\textbf{Núcleos Separables de Filtros}}
        \item \href{run:Tema_3_4_pag_166_p1.html}{\textbf{Lowpass Gaussian Filter Kernels}}
        \item \href{run:Tema_3_4_pag_169_p1.html}{\textbf{Ejemplos: Lowpass Gaussian Filter Kernels}}
        \item \href{run:Tema_3_4_pag_174_p1.html}{\textbf{Filtros de Estadística de Orden}}
        \item \href{run:Tema_3_6_p1.html}{\textbf{Filtros Espaciales de Realce (Pasa Altas)}}
        \item \href{run:Tema_3_6_pag_182.html}{\textbf{Enmascaramiento Desenfocado y High-Boost}}
        \item \href{run:Tema_3_6_pag_184.html}{\textbf{Derivadas de Primer Orden: El Gradiente}}
        \item \href{run:Tema_3_7_pag_188.html}{\textbf{Filtros de Señal: Highpass, Bandreject y Bandpass}}
        \item \href{run:Tema_3_8_pag_191.html}{\textbf{Métodos de Mejora Espacial en Procesamiento de Imágenes}}
    \end{itemize}
\end{frame}

\end{document}
